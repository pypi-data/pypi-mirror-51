%% $Date: 2012-07-06 17:42:54#$
%% $Revision: 152 $
\index{planck\_emission}
\algdesc{Planck Emission}
{ %%%%%% Algorithm name %%%%%%
planck\_emission
}
{ %%%%%% Algorithm summary %%%%%%
Calculates the Planck emission of a surface at a given wavelength given its temperature.
}
{ %%%%%% Category %%%%%%
Radiation
}
{ %%%%%% Inputs %%%%%%
$T$ & Vector & Temperature [K] \\
$\lambda$ & Coeff & Wavelength [nm] \\
}
{ %%%%%% Outputs %%%%%%
$rad$ & Vector & Black body radiance [W m-2 sr-1 nm-1] \\
}
{ %%%%%% Formula %%%%%%
After converting $\lambda$ to meters, the radiance is calculated by:
%
\begin{displaymath}
rad = \frac{2 h c^2}{\lambda^5 (\exp(\frac{h c}{k_B \lambda T}) - 1)} * 10^{-9}
\end{displaymath}

where $c$ is the speed of light in m/s, $h$ is the Planck constant in J s and $k_B$ is the Boltzmann
constant in J/K.

}
{ %%%%%% Author %%%%%%
Andre Ehrlich, Leipzig Institute for Meteorology (a.ehrlich@uni-leipzig.de)
}
{ %%%%%% References %%%%%% 

}


