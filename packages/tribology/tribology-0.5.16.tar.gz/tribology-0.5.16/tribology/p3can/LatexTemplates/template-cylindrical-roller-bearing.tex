
\clearpage
\pagebreak
\subsection{Contact Geometry}
\begin{figure}[h!]
  \centering
  \subfloat[Contact roller and inner ring]{\includegraphics[width=0.49\textwidth]{\VAR{contact_plot1}}
  }
  \hfill
  \subfloat[Contact roller and outer ring]{\includegraphics[width=0.49\textwidth]{\VAR{contact_plot2}}
  }
  \caption{Contact plots.}
\end{figure}

\pagebreak
\subsection{Kinematics}
\begin{figure}[h!]
  \centering
    \includegraphics[width=0.45\textwidth]{\VAR{slip_plot1}}
    \caption{Roller slip (cage rotates counterclockwise)}
\end{figure}

\pagebreak
\section{Simulation Results}
\subsection{Results Summary}
\begin{tabular}{ l l l l}
variable & unit & value & status \\
\midrule
\BLOCK{ for key, value, unit, status in table_calc_summary}
\VAR{key}  & \VAR{unit} & \VAR{value} & \VAR{status} \\
\BLOCK{ endfor }
\end{tabular}
\pagebreak
\clearpage

\subsection{Load Distribution}
\begin{figure}[h!]
  \centering
    \includegraphics[width=0.45\textwidth]{\VAR{load_plot1}}
    \caption{Load distribution for contact with highest normal force.}
\end{figure}

\subsection{Contact Pressure}
\begin{figure}[h!]
  \centering
  \subfloat[Contact roller and inner ring]{\includegraphics[width=0.49\textwidth]{\VAR{pressure_plot1}}
  }
  \hfill
  \subfloat[Contact roller and outer ring]{\includegraphics[width=0.49\textwidth]{\VAR{pressure_plot2}}
  }
  \caption{Contact pressure at highest normal force.}
\end{figure}

\pagebreak
\subsection{Energy Figures}
\begin{figure}[!h]
\centering
  \includegraphics[width=0.5\textwidth]{\VAR{energy_plot1}}
  \caption{pv\textsubscript{rel,max} inner and outer ring.}
\end{figure}

\begin{figure}[!h]
\centering
  \includegraphics[width=\VAR{scale_factor_or}\textwidth]{\VAR{energy_plot2}}
  \caption{e\textsubscript{a,kin} inner ring}
\end{figure}

\begin{figure}[!h]
\centering
  \includegraphics[width=\VAR{scale_factor_ir}\textwidth]{\VAR{energy_plot3}}
  \caption{e\textsubscript{a,kin} outer ring.}
\end{figure}
